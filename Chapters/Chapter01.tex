%************************************************
\chapter{Introduzione}\label{ch:introduzione}
%************************************************
\section{Reattori nucleari}

I reattori nucleari sono nati negli anni '50. Il primo reattore nucleare sperimentale a essere collegato a una rete elettrica fu l'impianto da 5 MW di Obninsk, acceso il 27 giugno 1954. Seguirono gli impianti commerciali di Calder Hall presso Sellafield in Inghilterra, nel 1956, con una capacit� di 50 MW, seguito un anno dopo dal primo reattore statunitense, l'impianto di Shippingport in Pennsylvania \cite{:kx}.

\section{La terza generazione, (gen3)}

i reattori di terza generazione

\section{La generazione intermedia (gen3+)}

i reattori di terza generazione evoluta

\section{La quarta generazione}

i reattori di quarta generazione


A very important factor for successful thesis writing is the
organization of the material. This template suggests a structure as
the following:
\begin{itemize}
    \graffito{You can use these margins for summaries of the text
    body\dots}
    \item\texttt{Chapters/} is where all the ``real'' content goes in
    separate files such as \texttt{Chapter01.tex} etc.
 %  \item\texttt{Examples/} is where you store all listings and other
 %  examples you want to use for your text.
    \item\texttt{FrontBackMatter/} is where all the stuff goes that
    surrounds the ``real'' content, such as the acknowledgments,
    dedication, etc.
    \item\texttt{gfx/} is where you put all the graphics you use in
    the thesis. Maybe they should be organized into subfolders
    depending on the chapter they are used in, if you have a lot of
    graphics.
    \item\texttt{Bibliography.bib}: the Bib\TeX\ database to organize
    all the references you might want to cite.
    \item\texttt{classicthesis.sty}: the style definition to get this
    awesome look and feel. 
    \item\texttt{ClassicThesis.tcp} a \TeX nicCenter project file.
    Great tool and it's free!
    \item\texttt{ClassicThesis.tex}: the main file of your thesis
    where all gets bundled together.
    \item\texttt{classicthesis-ldpkg.sty}: a central place to load all 
    nifty packages that are used. The package has the following options 
    available:
		\begin{itemize}
			\item\texttt{nochapters}, which defaults to \texttt{false}.
		    Activate it if you want to use the package with a class which does
		    not have chapter divisions, \eg, an article.
		    \item\texttt{backref}, which also defaults to \texttt{false}.
		    Activate it if you do want to show in the bibliography on which
		    page(s) each reference was cited. %See page~\pageref{app:bibliography} 
		    for an example of the default setting.
		\end{itemize}
\end{itemize}
This should get you started in no time.


\section{Style Options}
There are a couple of options for \texttt{classicthesis.sty} that
allow for a bit of freedom concerning the layout:
\begin{itemize}
    \graffito{\dots or your supervisor might use the margins for some
    comments of her own while reading.}
    \item\texttt{drafting}: prints the date and time at the bottom of
    each page, so you always know which version you are dealing with.
    Might come in handy not to give your Prof. that old draft.
    \item\texttt{eulerchapternumbers}: use figures from Hermann Zapf's
    Euler math font for the chapter numbers. By default, old style
    figures from the Palatino font are used.
    \item\texttt{linedheaders}: changes the look of the chapter
    headings a bit by adding a horizontal line above the chapter
    title. The chapter number will also be moved to the top of the
    page, above the chapter title.
    \item\texttt{listsseparated}: will add extra space between table
    and figure entries of different chapters in the list of tables or
    figures, respectively.
    \item\texttt{tocaligned}: aligns the whole table of contents on
    the left side. Some people like that, some don't.
    \item\texttt{subfig}(\texttt{ure}): is passed to the \texttt{tocloft} 
    package to enable compatibility with the \texttt{subfig}(\texttt{ure}) 
    package.
    \item\texttt{nochapters}: allows to use the look-and-feel with 
    classes that do not use chapters, \eg, for articles. Automatically
    turns off a couple of other options: \texttt{eulerchapternumbers}, 
    \texttt{linedheaders}, \texttt{listsseparated}, and \texttt{parts}.
    \item\texttt{beramono}: loads Bera Mono as typewriter font. 
    (Default setting is using the standard CM typewriter font.)
    \item\texttt{eulermath}: loads the awesome Euler fonts for math. 
    (Palatino is used as default font.)
    \item\texttt{parts}: if you use Part divisions for your document,
    you should choose this option. It provides you with the command
    \verb|\myPart{}| which takes care of the style and the entry
    into the Table of Contents. (Cannot be used together with
    \texttt{nochapters}.)
    \item\texttt{a5paper}: adjusts the page layout according to the
    global \texttt{a5paper} option (\emph{experimental} feature).
    \item\texttt{minionpro}: sets Robert Slimbach's Minion as the 
    main font of the document. The textblock size is adjusted 
    accordingly. 
    \item\texttt{pdfspacing}: makes use of pdftex' letter spacing
    capabilities via the \texttt{microtype} package.\footnote{Use 
    \texttt{microtype}'s \texttt{DVIoutput} option to generate
    DVI with pdftex.} This fixes some serious issues regarding 
    math formul\ae\ etc. (\eg, ``\ss'') in headers. 
    \item\texttt{minionprospacing}: uses the internal \texttt{textssc}
    command of the \texttt{MinionPro} package for letter spacing. This 
    automatically enables the \texttt{minionpro} option and overrides
    the \texttt{pdfspacing} option.
    \item\texttt{dottedtoc}: sets pagenumbers flushed right in the 
    table of contents.
    \item\texttt{listings}: loads the \texttt{listings} package (if not 
    already done) and configures the List of Listings accordingly.
    \item\texttt{manychapters}: if you need more than nine chapters for 
    your document, you might not be happy with the spacing between the 
    chapter number and the chapter title in the Table of Contents. 
    This option allows for additional space in this context. 
    However, it does not look as ``perfect'' if you use
    \verb|\parts| for structuring your document.
\end{itemize}
The best way to figure these options out is to try the different
possibilities and see, what you and your supervisor like best.

To make things in general easier, \texttt{classicthesis-ldpkg.sty} 
contains some useful commands that might help you.


\section{Future Work}
So far, this is a quite stable version that served a couple of people
well during their thesis time. However, some things are still not as
they should be. Proper documentation in the standard format is still
missing. In the long run, the style should probably be published
separately, with the template bundle being only an application of the
style. Alas, there is no time for that at the moment\dots it could be
a nice task for a small group of \LaTeX nicians.

Please do not send me email with questions concerning \LaTeX\ or the
template, as I do not have time for an answer. But if you have
comments, suggestions, or improvements for the style or the template
in general, do not hesitate to write them on that postcard of yours.


\section{License}
\paragraph{GNU General Public License:} This program is free software;
you can redistribute it and/or modify
 it under the terms of the \textsmaller{GNU} General Public License as
 published by
 the Free Software Foundation; either version 2 of the License, or
 (at your option) any later version.

 This program is distributed in the hope that it will be useful,
 but \emph{without any warranty}; without even the implied warranty of
 \emph{merchantability} or \emph{fitness for a particular purpose}.
 See the
 \textsmaller{GNU} General Public License for more details.

 You should have received a copy of the \textsmaller{GNU} General
 Public License
 along with this program; see the file \texttt{COPYING}.  If not,
 write to
 the Free Software Foundation, Inc., 59 Temple Place - Suite 330,
 Boston, \textsmaller{MA} 02111-1307, \textsmaller{USA}.


\section{Beyond a Thesis}
It is easy to use the layout of \texttt{classicthesis.sty} without the
framework of this bundle. To make it even easier, this section offers 
some plug-and-play-examples.

The \LaTeX -sources of these examples can be found in the folder 
with the name \texttt{Examples}. They have been tested with  
\texttt{latex} and \texttt{pdflatex} and are easy to compile. To 
assure you even a bit more, PDFs built from the sources can also 
be found the folder. 
%(It might be necessary to adjust the path to 
%\texttt{classicthesis.sty} and \texttt{Bibliography.bib} within the 
%examples.)

\lstinputlisting[caption=An Article]%
    {Examples/classicthesis-article.tex}
    
\lstinputlisting[caption=A Book]%
    {Examples/classicthesis-book.tex}

\lstinputlisting[caption=A Curriculum Vit\ae]%
    {Examples/classicthesis-cv.tex}

%*****************************************
%*****************************************
%*****************************************
%*****************************************
%*****************************************




