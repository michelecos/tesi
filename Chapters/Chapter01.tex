%************************************************
\chapter{Introduzione}
%************************************************
\section{Reattori nucleari}

I reattori nucleari sono nati negli anni '50. Il primo reattore nucleare sperimentale a essere collegato a una rete elettrica fu l'impianto da 5 MW di Obninsk, acceso il 27 giugno 1954. Seguirono gli impianti commerciali di Calder Hall \cite{:hc} presso Sellafield in Inghilterra, nel 1956, con una capacit� di 50 MW, seguito un anno dopo dal primo reattore statunitense, l'impianto di Shippingport in Pennsylvania \cite{:qf}.

Il Dipartimento Dell'Energia degli Stati Uniti (DOE) classifica i reattori nucleari in quattro generazioni principali e una intermedia, quindi parliamo di reattori di generazione I, II, III, III+ e IV \cite{:kl}.

La prima generazione di reattori comprende modelli sperimentali, come l'impianto di Shippingport, Ohio \cite{:qf}, il progetto inglese Magnox \cite{:bh}, l'impianto Fermi 1 sul lago Erie nel Michigan \cite{:dq}, e la centrale di Dresden, Illinois \cite{:cr}.

% vedi http://en.wikipedia.org/wiki/Generation_II_reactor
I progetti di seconda generazione sono PWR, CANDU, BWR, AGR, and VVER.[1]

% vedi http://en.wikipedia.org/wiki/Generation_III_reactor
La terza generazione di reattori porta miglioramenti significativi all'efficienza ABWR APWR CANDU (EC6) VVER




\section{La terza generazione, (gen3)}

i reattori di terza generazione

\subsection{}


\section{La generazione intermedia (gen3+)}

i reattori di terza generazione evoluta

\section{La quarta generazione}

i reattori di quarta generazione






