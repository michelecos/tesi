%************************************************
\chapter{Introduzione}
%************************************************

Nel 1938 Enrico Fermi riceve il premio Nobel per la fisica per i suoi studi sulla radioattivit� artificiale, cio� la fissione indotta con un bombardamento di neutroni. Nel discorso di accettazione \cite{:fy} espone una scoperta fondamentale: un fascio di neutroni rallentato attraverso le collisioni con un materiale ricco di idrogeno, come la paraffina o l'acqua, interagisce con i nuclei di un materiale radioattivo in misura molto maggiore.

\begin{quotation}
The intensity of the activation as a function of the distance from the neutron source shows in some cases anomalies apparently dependent on the objects that surround the source. A careful investigation of these effects led to the unexpected result that surrounding both source and body to be activated with masses of paraffin, increases in some cases the intensity of activation by a very large factor (up to 100). A similar effect is produced by water, and in general by substances containing a large concentration of hydrogen. Substances not containing hydrogen show sometimes similar features, though extremely less pronounced.
\end{quotation}

Il secondo tassello arriva un anno dopo. In una lettera a Szilard\cite{:lsz} Fermi mostra di intuire la potenzialit� di una reazione a catena e identifica l'acqua pesante come un possibile moderatore per ridurre la velocit� dei neutroni emessi dalla fissione di una massa critica di Uranio.

Nel 1939, Fermi realizza a Chicago il primo prototipo di reattore nucleare, la pila atomica, aprendo la possibilit� di creare enormi quantit� di energia con minuscole quantit� di combustibile.

Diversi anni dopo, l'energia nucleare torna in cima alla lista nell'agenda energetica dei governi, nonostante il marchio della paura, causato dalle bombe di Hiroshima e Nagasaki e dalla catastrofe di Chernobyl \cite{:chr}. Il mondo, infatti, deve affrontare i problemi posti dalla possibilit� di una crisi nell'estrazione del petrolio entro venti anni \cite{:uq}, e dalla necessit� di limitare le emissioni di CO2 per frenare il riscaldamento globale. Difficile dare una risposta a queste domande, senza valutare l'installazione di nuove centrali.

In questa tesi parleremo di tecnologie mature, quelle su cui il mondo pu� puntare nel breve periodo, discutendo la struttura delle realizzazioni di \ac{GEN III} e \ac{GEN III+}. Restringeremo il campo di interesse alla struttura delle diverse filiere e ai diversi tipi di combustibile che le alimentano.

La tesi, quindi, si aprir� con un'introduzione alla fisica e all'ingegneria dei reattori, nel secondo capitolo passeremo in rassegna i principi costruttivi delle diverse filiere e nel terzo capitolo esamineremo i combustibili che possono alimentare le diverse filiere, accennando ai processi di fabbricazione.